% In this file you should put the actual content of the blueprint.
% It will be used both by the web and the print version.
% It should *not* include the \begin{document}
%
% If you want to split the blueprint content into several files then
% the current file can be a simple sequence of \input. Otherwise It
% can start with a \section or \chapter for instance.

\documentclass{amsart}
\usepackage{graphicx} % Required for inserting images
\usepackage{fullpage}
\usepackage{amsthm,amsmath,amssymb,amsfonts}
\usepackage{hyperref}
\usepackage{tikz-cd}

\newtheorem{theorem}{Theorem}
\numberwithin{theorem}{subsection}
\newtheorem{lemma}[theorem]{Lemma}
\newtheorem{corollary}[theorem]{Corollary}
\newtheorem{conjecture}[theorem]{Conjecture}

\newtheorem{proposition}[theorem]{Proposition}
\theoremstyle{definition}
\newtheorem{construction}[theorem]{Construction}
\newtheorem{definition}[theorem]{Definition}
\newtheorem{hypothesis}[theorem]{Hypothesis}
\newtheorem{example}[theorem]{Example}
\newtheorem{remark}[theorem]{Remark}
\newtheorem{convention}[theorem]{Convention}
\newtheorem{notation}[theorem]{Notation}
\newtheorem{property}[theorem]{Property}

\numberwithin{equation}{subsection}

\newcommand{\Kos}{\mathrm{Kos}}

\def\calA{\mathcal{A}}
\def\calB{\mathcal{B}}
\def\calC{\mathcal{C}}
\def\calD{\mathcal{D}}
\def\calE{\mathcal{E}}
\def\calF{\mathcal{F}}
\def\calG{\mathcal{G}}
\def\calH{\mathcal{H}}
\def\calI{\mathcal{I}}
\def\calJ{\mathcal{J}}
\def\calK{\mathcal{K}}
\def\calL{\mathcal{L}}
\def\calM{\mathcal{M}}
\def\calN{\mathcal{N}}
\def\calO{\mathcal{O}}
\def\calP{\mathcal{P}}
\def\calQ{\mathcal{Q}}
\def\calR{\mathcal{R}}
\def\calS{\mathcal{S}}
\def\calT{\mathcal{T}}
\def\calU{\mathcal{U}}
\def\calV{\mathcal{V}}
\def\calW{\mathcal{W}}
\def\calX{\mathcal{X}}
\def\calY{\mathcal{Y}}
\def\calZ{\mathcal{Z}}

\def\frakA{\mathfrak{A}}
\def\frakB{\mathfrak{B}}
\def\frakC{\mathfrak{C}}
\def\frakD{\mathfrak{D}}
\def\frakE{\mathfrak{E}}
\def\frakF{\mathfrak{F}}
\def\frakG{\mathfrak{G}}
\def\frakH{\mathfrak{H}}
\def\frakI{\mathfrak{I}}
\def\frakJ{\mathfrak{J}}
\def\frakK{\mathfrak{K}}
\def\frakL{\mathfrak{L}}
\def\frakM{\mathfrak{M}}
\def\frakN{\mathfrak{N}}
\def\frakO{\mathfrak{O}}
\def\frakP{\mathfrak{P}}
\def\frakQ{\mathfrak{Q}}
\def\frakR{\mathfrak{R}}
\def\frakS{\mathfrak{S}}
\def\frakT{\mathfrak{T}}
\def\frakU{\mathfrak{U}}
\def\frakV{\mathfrak{V}}
\def\frakW{\mathfrak{W}}
\def\frakX{\mathfrak{X}}
\def\frakY{\mathfrak{Y}}
\def\frakZ{\mathfrak{Z}}
\def\fraka{\mathfrak{a}}
\def\frakb{\mathfrak{b}}
\def\frakc{\mathfrak{c}}
\def\frakd{\mathfrak{d}}
\def\frake{\mathfrak{e}}
\def\frakf{\mathfrak{f}}
\def\frakg{\mathfrak{g}}
\def\frakh{\mathfrak{h}}
\def\fraki{\mathfrak{i}}
\def\frakj{\mathfrak{j}}
\def\frakk{\mathfrak{k}}
\def\frakl{\mathfrak{l}}
\def\frakm{\mathfrak{m}}
\def\frakn{\mathfrak{n}}
\def\frako{\mathfrak{o}}
\def\frakp{\mathfrak{p}}
\def\frakq{\mathfrak{q}}
\def\frakr{\mathfrak{r}}
\def\fraks{\mathfrak{s}}
\def\frakt{\mathfrak{t}}
\def\fraku{\mathfrak{u}}
\def\frakv{\mathfrak{v}}
\def\frakw{\mathfrak{w}}
\def\frakx{\mathfrak{x}}
\def\fraky{\mathfrak{y}}
\def\frakz{\mathfrak{z}}
\def\AAA{\mathbb{A}}
\def\BB{\mathbb{B}}
\def\CC{\mathbb{C}}
\def\DD{\mathbb{D}}
\def\EE{\mathbb{E}}
\def\FF{\mathbb{F}}
\def\GG{\mathbb{G}}
\def\HH{\mathbb{H}}
\def\II{\mathbb{I}}
\def\JJ{\mathbb{J}}
\def\KK{\mathbb{K}}
\def\LL{\mathbb{L}}
\def\MM{\mathbb{M}}
\def\NN{\mathbb{N}}
\def\OO{\mathbb{O}}
\def\PP{\mathbb{P}}
\def\QQ{\mathbb{Q}}
\def\RR{\mathbb{R}}
\def\SSS{\mathbb{S}}
\def\TT{\mathbb{T}}
\def\UU{\mathbb{U}}
\def\VV{\mathbb{V}}
\def\WW{\mathbb{W}}
\def\XX{\mathbb{X}}
\def\YY{\mathbb{Y}}
\def\ZZ{\mathbb{Z}}
\def\bfa{\mathbf{a}}
\def\bfb{\mathbf{b}}
\def\bfc{\mathbf{c}}
\def\bfd{\mathbf{d}}
\def\bfe{\mathbf{e}}
\def\bff{\mathbf{f}}
\def\bfg{\mathbf{g}}
\def\bfh{\mathbf{h}}
\def\bfi{\mathbf{i}}
\def\bfj{\mathbf{j}}
\def\bfk{\mathbf{k}}
\def\bfl{\mathbf{l}}
\def\bfm{\mathbf{m}}
\def\bfn{\mathbf{n}}
\def\bfo{\mathbf{o}}
\def\bfp{\mathbf{p}}
\def\bfq{\mathbf{q}}
\def\bfr{\mathbf{r}}
\def\bfs{\mathbf{s}}
\def\bft{\mathbf{t}}
\def\bfu{\mathbf{u}}
\def\bfv{\mathbf{v}}
\def\bfw{\mathbf{w}}
\def\bfx{\mathbf{x}}
\def\bfy{\mathbf{y}}
\def\bfz{\mathbf{z}}
\def\bfA{\mathbf{A}}
\def\bfB{\mathbf{B}}
\def\bfC{\mathbf{C}}
\def\bfD{\mathbf{D}}
\def\bfE{\mathbf{E}}
\def\bfF{\mathbf{F}}
\def\bfG{\mathbf{G}}
\def\bfH{\mathbf{H}}
\def\bfI{\mathbf{I}}
\def\bfJ{\mathbf{J}}
\def\bfK{\mathbf{K}}
\def\bfL{\mathbf{L}}
\def\bfM{\mathbf{M}}
\def\bfN{\mathbf{N}}
\def\bfO{\mathbf{O}}
\def\bfP{\mathbf{P}}
\def\bfQ{\mathbf{Q}}
\def\bfR{\mathbf{R}}
\def\bfS{\mathbf{S}}
\def\bfT{\mathbf{T}}
\def\bfU{\mathbf{U}}
\def\bfV{\mathbf{V}}
\def\bfW{\mathbf{W}}
\def\bfX{\mathbf{X}}
\def\bfY{\mathbf{Y}}
\def\bfZ{\mathbf{Z}}
\def\rma{\mathrm{a}}
\def\rmb{\mathrm{b}}
\def\rmc{\mathrm{c}}
\def\rmd{\mathrm{d}}
\def\rme{\mathrm{e}}
\def\rmf{\mathrm{f}}
\def\rmg{\mathrm{g}}
\def\rmh{\mathrm{h}}
\def\rmi{\mathrm{i}}
\def\rmj{\mathrm{j}}
\def\rmk{\mathrm{k}}
\def\rml{\mathrm{l}}
\def\rmm{\mathrm{m}}
\def\rmn{\mathrm{n}}
\def\rmo{\mathrm{o}}
\def\rmp{\mathrm{p}}
\def\rmq{\mathrm{q}}
\def\rmr{\mathrm{r}}
\def\rms{\mathrm{s}}
\def\rmt{\mathrm{t}}
\def\rmu{\mathrm{u}}
\def\rmv{\mathrm{v}}
\def\rmw{\mathrm{w}}
\def\rmx{\mathrm{x}}
\def\rmy{\mathrm{y}}
\def\rmz{\mathrm{z}}
\def\rmA{\mathrm{A}}
\def\rmB{\mathrm{B}}
\def\rmC{\mathrm{C}}
\def\rmD{\mathrm{D}}
\def\rmE{\mathrm{E}}
\def\rmF{\mathrm{F}}
\def\rmG{\mathrm{G}}
\def\rmH{\mathrm{H}}
\def\rmI{\mathrm{I}}
\def\rmJ{\mathrm{J}}
\def\rmK{\mathrm{K}}
\def\rmL{\mathrm{L}}
\def\rmM{\mathrm{M}}
\def\rmN{\mathrm{N}}
\def\rmO{\mathrm{O}}
\def\rmP{\mathrm{P}}
\def\rmQ{\mathrm{Q}}
\def\rmR{\mathrm{R}}
\def\rmS{\mathrm{S}}
\def\rmT{\mathrm{T}}
\def\rmU{\mathrm{U}}
\def\rmV{\mathrm{V}}
\def\rmW{\mathrm{W}}
\def\rmX{\mathrm{X}}
\def\rmY{\mathrm{Y}}
\def\rmZ{\mathrm{Z}}
\def\ttA{\mathtt{A}}
\def\ttB{\mathtt{B}}
\def\ttC{\mathtt{C}}
\def\ttD{\mathtt{D}}
\def\ttE{\mathtt{E}}
\def\ttF{\mathtt{F}}
\def\ttG{\mathtt{G}}
\def\ttH{\mathtt{H}}
\def\ttI{\mathtt{I}}
\def\ttJ{\mathtt{J}}
\def\ttK{\mathtt{K}}
\def\ttL{\mathtt{L}}
\def\ttM{\mathtt{M}}
\def\ttN{\mathtt{N}}
\def\ttO{\mathtt{O}}
\def\ttP{\mathtt{P}}
\def\ttQ{\mathtt{Q}}
\def\ttR{\mathtt{R}}
\def\ttS{\mathtt{S}}
\def\ttT{\mathtt{T}}
\def\ttU{\mathtt{U}}
\def\ttV{\mathtt{V}}
\def\ttW{\mathtt{W}}
\def\ttX{\mathtt{X}}
\def\ttY{\mathtt{Y}}
\def\ttZ{\mathtt{Z}}
\def\tta{\mathtt{a}}
\def\ttb{\mathtt{b}}
\def\ttc{\mathtt{c}}
\def\ttd{\mathtt{d}}
\def\tte{\mathtt{e}}
\def\ttf{\mathtt{f}}
\def\ttg{\mathtt{g}}
\def\tth{\mathtt{h}}
\def\tti{\mathtt{i}}
\def\ttj{\mathtt{j}}
\def\ttk{\mathtt{k}}
\def\ttl{\mathtt{l}}
\def\ttm{\mathtt{m}}
\def\ttn{\mathtt{n}}
\def\tto{\mathtt{o}}
\def\ttp{\mathtt{p}}
\def\ttq{\mathtt{q}}
\def\ttr{\mathtt{r}}
\def\tts{\mathtt{s}}
\def\ttt{\mathtt{t}}
\def\ttu{\mathtt{u}}
\def\ttv{\mathtt{v}}
\def\ttw{\mathtt{w}}
\def\ttx{\mathtt{x}}
\def\tty{\mathtt{y}}
\def\ttz{\mathtt{z}}
\def\scrA{\mathscr{A}}
\def\scrB{\mathscr{B}}
\def\scrC{\mathscr{C}}
\def\scrD{\mathscr{D}}
\def\scrE{\mathscr{E}}
\def\scrF{\mathscr{F}}
\def\scrG{\mathscr{G}}
\def\scrH{\mathscr{H}}
\def\scrI{\mathscr{I}}
\def\scrJ{\mathscr{J}}
\def\scrK{\mathscr{K}}
\def\scrL{\mathscr{L}}
\def\scrM{\mathscr{M}}
\def\scrN{\mathscr{N}}
\def\scrO{\mathscr{O}}
\def\scrP{\mathscr{P}}
\def\scrQ{\mathscr{Q}}
\def\scrR{\mathscr{R}}
\def\scrS{\mathscr{S}}
\def\scrT{\mathscr{T}}
\def\scrU{\mathscr{U}}
\def\scrV{\mathscr{V}}
\def\scrW{\mathscr{W}}
\def\scrX{\mathscr{X}}
\def\scrY{\mathscr{Y}}
\def\scrZ{\mathscr{Z}}
\def\i{\mathbf{i}}
\def\j{\mathbf{j}}
\def\k{\mathbf{k}}
\def\sfa{\mathsf{a}}
\def\sfb{\mathsf{b}}
\def\sfc{\mathsf{c}}
\def\sfd{\mathsf{d}}
\def\sfe{\mathsf{e}}
\def\sff{\mathsf{f}}
\def\sfg{\mathsf{g}}
\def\sfh{\mathsf{h}}
\def\sfi{\mathsf{i}}
\def\sfj{\mathsf{j}}
\def\sfk{\mathsf{k}}
\def\sfl{\mathsf{l}}
\def\sfm{\mathsf{m}}
\def\sfn{\mathsf{n}}
\def\sfo{\mathsf{o}}
\def\sfp{\mathsf{p}}
\def\sfq{\mathsf{q}}
\def\sfr{\mathsf{r}}
\def\sfs{\mathsf{s}}
\def\sft{\mathsf{t}}
\def\sfu{\mathsf{u}}
\def\sfv{\mathsf{v}}
\def\sfw{\mathsf{w}}
\def\sfx{\mathsf{x}}
\def\sfy{\mathsf{y}}
\def\sfz{\mathsf{z}}
\def\sfA{\mathsf{A}}
\def\sfB{\mathsf{B}}
\def\sfC{\mathsf{C}}
\def\sfD{\mathsf{D}}
\def\sfE{\mathsf{E}}
\def\sfF{\mathsf{F}}
\def\sfG{\mathsf{G}}
\def\sfH{\mathsf{H}}
\def\sfI{\mathsf{I}}
\def\sfJ{\mathsf{J}}
\def\sfK{\mathsf{K}}
\def\sfL{\mathsf{L}}
\def\sfM{\mathsf{M}}
\def\sfN{\mathsf{N}}
\def\sfO{\mathsf{O}}
\def\sfP{\mathsf{P}}
\def\sfQ{\mathsf{Q}}
\def\sfR{\mathsf{R}}
\def\sfS{\mathsf{S}}
\def\sfT{\mathsf{T}}
\def\sfU{\mathsf{U}}
\def\sfV{\mathsf{V}}
\def\sfW{\mathsf{W}}
\def\sfX{\mathsf{X}}
\def\sfY{\mathsf{Y}}
\def\sfZ{\mathsf{Z}}

\newcommand{\Fil}{\mathrm{Fil}}
\newcommand{\Gr}{\mathrm{Gr}}
\newcommand{\Rees}{\mathrm{Ree}}
\DeclareMathOperator{\supp}{Supp}
\DeclareMathOperator{\Ker}{Ker}
\DeclareMathOperator{\Ext}{Ext}
\DeclareMathOperator{\Ann}{Ann}
\DeclareMathOperator{\Hom}{Hom}
\DeclareMathOperator{\Tor}{Tor}
\DeclareMathOperator{\depth}{depth}
\DeclareMathOperator{\Ass}{Ass}

\title{Cohen-Macaulay}
\author{AI4M team}
\begin{document}
	
	\maketitle
	
	
	\section{Regular sequence}
	
	\subsection{Definitions and basic properties}
	
	\begin{definition}
		Let $R$ be a ring and $M$ an $R$-module. An element $x \in R$ is called \emph{$M$-regular} if $x \cdot : M \to M$ is injective. 
		
		(Equivalent definition: if for any $m \in M$, $xm=0$ implies $m=0$.) 
		
		(Alternative wording: $x$ is not a zero-divisor on $M$.)
		
		(Caution: if $M$ is the zero module, any element is a regular element.
		
		Before \cite[Definition 1.1.1]{BH}.
	\end{definition}
	
	\begin{definition}
		A sequence $\bfx  = (x_1, \dots, x_n)$ of elements of $R$ is called a \emph{$M$-regular sequence} if inductively
		\begin{itemize}
			\item $x_i$ is an $M/(x_1, \dots, x_{i-1})M$-regular element for each $i=1, \dots, n$;
			\item $M/ (x_1, \dots, x_n)M \neq 0$.
		\end{itemize}
		A \emph{weak $M$-sequence} is a sequence $\bfx$ satisfying the first condition.
		
		\cite[Definition 1.1.1]{BH}. \cite[(15.A)]{matsumura-ring}.
		
	\end{definition}
	
	\begin{remark}
		If $R$ is a local ring with maximal ideal $\frakm$, then if $\bfx \subset \frakm$, then if $\bfx$ is weak $M$-sequence, then it is a $M$-regular sequence (by Nakayama's lemma).
		
		After \cite[Definition 1.1.1]{BH}.
	\end{remark}
	
	\begin{lemma}
		\label{lemma-join-regular-sequences}
		Let $A$ be a ring. Let $I$ be an ideal generated by a regular
		sequence $f_1, \ldots, f_n$ in $A$. Let $g_1, \ldots, g_m \in A$ be
		elements whose images $\overline{g}_1, \ldots, \overline{g}_m$ form a
		regular sequence in $A/I$. Then $f_1, \ldots, f_n, g_1, \ldots, g_m$
		is a regular sequence in $A$.
		
		\cite[Lemma 065K]{stacks-project}.
	\end{lemma}
	
	\begin{proof}
		This follows immediately from the definitions.
	\end{proof}
	
	The following seems to be very powerful technical tools to prove results regarding regular sequences.
	
	\begin{proposition}\label{exact_sequence}
		Let $R$ be a ring, $M$ an $R$-module, and $\bfx$ a weak $M$-sequence. Then an exact sequence
		$$
		N_2 \xrightarrow{\varphi_2} N_1 \xrightarrow{\varphi_1}N_0 \xrightarrow{\varphi_0}M \to 0
		$$
		of $R$-modules induces an exact sequence
		$$
		N_2/\bfx N_2 \to N_1 /\bfx N_1 \to N_0 /\bfx N_0 \to M / \bfx M \to 0.
		$$
		
		\cite[Proposition 1.1.4]{BH}.
	\end{proposition}
	
	\begin{proof}
		By induction it is enough to consider the case in which $\bfx$ consists of a single $M$-regular element $x$. We obtain the induced sequence if we tensor the original one by $R/(x)$. Since tensor product is a right exact functor, we only need to verify exactness at $N_1/xN_1$. Let $^-$ denote residue classes modulo $x$. If $\bar{\varphi}_1(\bar{y}) = 0$, then $\varphi_1(y) = xz$ for some $z \in N_0$ and $x\varphi_0(z) = 0$. By hypothesis we have $\varphi_0(z) = 0$; hence there is $y' \in N_1$ with $z = \varphi_1(y')$. It follows that $\varphi_1(y - xy') = 0$. So $y - xy' \in \varphi_2(N_2)$, and $\bar{y} \in \bar{\varphi}_2(\bar{N}_2)$ as desired.
	\end{proof}
	
	\begin{proposition}
		Let $R$ be a ring and
		$$
		N_\bullet: \ \cdots \to N_m \xrightarrow{\varphi_m} N_{m-1} \to \cdots \to N_0 \xrightarrow{\varphi_0}N_{-1} \to 0
		$$
		an exact complex of $R$-modules. If $\bfx$ is a weakly $N_i$-regular sequence for all $i$, then $N_\bullet \otimes R/(\bfx)$ is exact again.
		
		Note: need to think about whether the complex has finite length or not.
		
		\cite[Proposition 1.1.5]{BH}.
	\end{proposition}
	
	\begin{proof}
		Once more one uses induction on the length of the sequence $\bfx$. So it is enough to treat the case $\bfx = x$. Since $x$ is regular on $N_i$, it is regular on $\operatorname{Im} \varphi_{i+1}$ too. Therefore we can apply proposition \ref{exact_sequence} to each exact sequence $N_{i+3} \to N_{i+2} \to N_{i+1} \to \operatorname{Im} \varphi_{i+1} \to 0$.
	\end{proof}
	
	\begin{lemma}\label{short-exact-sequence}
		Let $0 \to M_1 \to M_2 \to M_3 \to 0$ be an exact sequence of $R$-modules, and $\bfx \in R$ is $M_1$-regular and $M_3$-regular. Then $\bfx$ is $M_2$-regular.
		
		\cite[Exercise 1.1.9]{BH}. \cite[\href{https://stacks.math.columbia.edu/tag/0F1T}{Lemma 0F1T}]{stacks-project}.
	\end{lemma}
	
	\begin{proof}
		By snake lemma \cite[\href{https://stacks.math.columbia.edu/tag/07JW}{Lemma 07JW}]{stacks-project}, if $f_1 : M_1 \to M_1$ and
		$f_1 : M_3 \to M_3$ are injective, then so is $f_1 : M_2 \to M_2$
		and we obtain a short exact sequence by lemma \ref{exact_sequence}
		\begin{center}
			\begin{tikzcd}
				0 \arrow[r] & M_1/x_1M_1 \arrow[r] & M_2/x_1M_2 \arrow[r] & M_3/x_1M_3 \arrow[r] & 0.
			\end{tikzcd}
		\end{center}
		The lemma follows from this and induction on $r$.
	\end{proof}
	
	
	\begin{lemma}\label{isRegular_mul}
		Let $x_1, \dots, x_i, \dots, x_n$ and $x_1, \dots, x'_i, \dots, x_n$ are (weakly) $M$-regular. Show that $x_1, \dots, x_ix'_i, \dots, x_n$ is (weakly) $M$-Regular.
		
		\cite[Exercise 1.1.10(a)]{BH}
	\end{lemma}
	(Hint: the essential case, one finds an exact sequence with $M/x_1x'_1M$ as the middle term.)
	
	\begin{proof}
		We can assume $ i = 1 $ after replacing $ M $ by $ M / (x_1, \dots, x_{i -1})M $. Note that $ x_1 $ is not a zero-divisor of $ M $, so we have a short exact sequence by lemma \ref{exact_sequence}
		\begin{center}
			\begin{tikzcd}
				0 \arrow[r] & M/x_1'M \arrow[r, "\cdot\ x_1"] & M/x_1x_1'M \arrow[r] & M/x_1M \arrow[r] & 0.
			\end{tikzcd}
		\end{center}
		Since $ x_2, \dots, x_n $ are $ M/x_1M $ and $ M/x_1'M $-regular sequences, so it is an $ M / x_1x_1'M $ -regular sequence by lemma \ref{short-exact-sequence}.
	\end{proof}
	
	\begin{lemma}
		If $x_1, \dots, x_n$ is a (weakly) $M$-regular, then for any positive integers $e_1, \dots, e_n$, $x_1^{e_1},\dots, x_n^{e_n}$ is (weakly) $M$-regular.
		
		\cite[Exercise 1.1.10(b)]{BH} 
		
		(\cite[page 96]{matsumura-alg} has a different proof)
	\end{lemma}
	
	\begin{proof}
		Use lemma \ref{isRegular_mul} and induction.
	\end{proof}
	
	\begin{proposition}
		\label{P:regular sequence functoriality}
		Let $\varphi: R \to S$ be a ring homomorphism, $M$ an $R$-module, and $\bfx \subset R$ a weak $M$-regular sequence. Suppose that $N$ is an $S$-module that is $R$-flat.  Then $\varphi(x) \subset S$ is a weak $M \otimes_R N$-regular sequence.
		
		\cite[Proposition 1.1.2]{BH}. The case $R = S$ is already in mathlib (\href{https://leanprover-community.github.io/mathlib4_docs/Mathlib/RingTheory/Regular/RegularSequence.html#RingTheory.Sequence.IsWeaklyRegular.isWeaklyRegular_lTensor}{IsWeaklyRegular.isWeaklyRegular\_lTensor}).
		
	\end{proposition}
	
	\begin{proof}
		Multiplication by $x_i$ is the same operation on $M \otimes N$ as multiplication by $\varphi(x_i)$; so it suffices to consider $x$. The homothety $x_1: M \to M$ is injective, and $x_1 \otimes N$ is injective too, because $N$ is flat. Now $x_1 \otimes N$ is just multiplication by $x_1$ on $M \otimes N$. So $x_1$ is an $(M \otimes N)$-regular element. Next we have $(M \otimes N)/x_1(M \otimes N) \cong (M/x_1 M) \otimes N$; an inductive argument will therefore complete the proof.
	\end{proof}
	
	\begin{example}
		This is a {\bf very important special case of above, we need to make sure the citation is easy enough.}) When $S = R_\frakp$ is a localization. Then $\bfx$ is weak $M$-regular implies that $\bfx \subset R_\frakp$ is weak $M_\frakp$-regular.
	\end{example}
	
	\begin{remark}
		We have the following weaker variant of above, can be stated as corollaries.
		\begin{enumerate}
			\item 
			Let $R$ be a ring, $M$ an $R$-module, and $\bfx \subset R$ a weak $M$-regular sequence. Let $N$ be a flat $R$-module. Then $\bfx$ is a weak $M \otimes_RN$-regular sequence. 
			\item
			Let $R$ be a ring, $M$ an $R$-module, and $\bfx \subset R$ an $M$-regular sequence. Let $N$ be a \emph{faithfully} flat $R$-module. Then $\bfx$ is an $M \otimes_RN$-regular sequence. 
		\end{enumerate}
	\end{remark}
	
	\begin{corollary}
		Let $R$ be a \emph{noetherian} ring, $M$ be a finite $R$-module, and $\bfx$ be a $M$-regular sequence. Suppose that a prime $\frakp \in \supp M$ contains $\bfx$. Then $\bfx$ (as a sequence in $R_\frakp$) is a $M_\frakp$-regular sequence.
		
		\cite[Corollary 1.1.3(1)]{BH}
	\end{corollary}
	
	\begin{proof}
		We have just show that $\bfx$ is a weak $M_\frakp$-regular sequence. By hypothesis $M_{\mathfrak{p}} \neq 0$, and Nakayama's lemma implies $M_{\mathfrak{p}} \neq \mathfrak{p}M_{\mathfrak{p}}$. \textit{A fortiori} we have $xM_{\mathfrak{p}} \neq M_{\mathfrak{p}}$.
	\end{proof}
	
	\begin{corollary}
		Let $(R, \frakm)$ be a \emph{noetherian} local ring, $M$ a finite $R$-module, and $\bfx$ a $M$-regular sequence.  Then $\bfx$ (as a sequence in the completion $\hat R$) is an $\hat M$-regular sequence.
		
		\cite[Corollary 1.1.3(2)]{BH}
	\end{corollary}
	
	\begin{proof}
		We have just show that $\bfx$ is a weak $\hat M$-regular sequence. Note that $\bfx \subset \frakm$, so $\bfx$ is a $\hat M$-regular sequence.
	\end{proof}
	
	\begin{lemma}
		\label{lemma-regular-sequence-in-neighbourhood}
		Let $R$ be a Noetherian ring. Let $M$ be a finite $R$-module.
		Let $\mathfrak p$ be a prime. Let $x_1, \ldots, x_r$ be a sequence
		in $R$ whose image in $R_{\mathfrak p}$ forms an $M_{\mathfrak p}$-regular
		sequence. Then there exists a $g \in R$, $g \not \in \mathfrak p$
		such that the image of $x_1, \ldots, x_r$ in $R_g$ forms
		an $M_g$-regular sequence.
		
		\cite[\href{https://stacks.math.columbia.edu/tag/061L}{Lemma 061L}]{stacks-project}
	\end{lemma}
	\begin{proof}
		Set
		$$
		K_i = \Ker\left(x_i : M/(x_1, \ldots, x_{i - 1})M \to
		M/(x_1, \ldots, x_{i - 1})M\right).
		$$
		This is a finite $R$-module whose localization at $\mathfrak p$ is
		zero by assumption. Hence there exists a $g \in R$, $g \not \in \mathfrak p$
		such that $(K_i)_g = 0$ for all $i = 1, \ldots, r$. This $g$ works.
	\end{proof}
	
	
	\begin{proposition}
		Let $R$ be a noetherian local ring, $M$ a finite $R$-module, and $\bfx$ a $M$-sequence. Then every permutation of $\bfx$ is an $M$-sequence.
		
		\cite[\href{https://stacks.math.columbia.edu/tag/00LJ}{Lemma 00LJ}]{stacks-project}.
		
		(Already in Mathlib)
		
	\end{proposition}
	(Remark: suffices to prove permuting $x_i$ and $x_{i+1}$ preserves regularity.)
	
	\begin{example}
		I think this example should be included somehow.
		
		Let $k$ be a field. In the ring $k[x,y,z]$, the sequence $x,y(1-x),z(1-x)$ is regular but the sequence $y(1-x), z(1-x), x)$ is not.
		
		\cite[\href{https://stacks.math.columbia.edu/tag/00LG}{Example 00LG}]{stacks-project}.
	\end{example}
	
	\begin{lemma}
		(a) Prove that if $\bfx$ is a weak $M$-sequence, then $\Tor^R_1(M, R/(\bfx)) = 0$.
		
		(b) Prove that if, in addition, $\bfx$ is a weak $R$-sequence, then $\Tor^R_i(M, R/(\bfx)) = 0$ for all $i \geq 1$.
	\end{lemma}
	
	
	\subsection{Some more on Rees algebra}
	Rees algebra has already implemented in Mathlib, briefly. (The aim of that is the Artin-Rees theorem.)  But it seems that what we need is slightly different.
	
	
	
	\begin{notation}
		\label{N:construction of M[X]}
		Let $R$ be a ring, $\bfX = (X_1, \dots, X_n)$ indeterminates over $R$, and $M$ an $R$-module. We write $M[\bfX]$ for $M \otimes_R R[\bfX]$, for the \emph{polynomials with coefficients in $M$} (it is a {\bf graded} module over $R[\bfX]$; somehow, we need this notation; we also need to show that it is graded, and one can talk about homogeneous elements.)
		
		This concept has the following property: if $x_1, \dots, x_n$ is a sequence of elements in $R$, we may consider evaluation map $R[\bfX] \to R$ given by $X_i \mapsto x_i$. This induces an $R$-module homomorphism
		$$
		M[\bfX] \to M, \qquad X_i \mapsto x_i.
		$$
	\end{notation}
	
	
	\begin{definition}
		We will work with decreasingly filtered ring, that is, a ring $R$ with a decreasing filtration $\Fil^iR$ $(i \in \ZZ)$ such that
		\begin{itemize}
			\item each $\Fil^iR$ is an abelian subgroup,
			\item $\Fil^iR \cdot \Fil^j R \subseteq \Fil^{i+j}R$, and
			\item we require $\Fil^0R = R$ (so the decreasing filtration forces $\Fil^{-i} R = \Fil^0R$ when $i \geq 0$).
		\end{itemize}
		
		For such a filtered ring, denote the \emph{Rees algebra} to be the subalgebra of $R[X]$ given by:
		\begin{align*}
			\Rees(\Fil^\bullet R)\ &: = \bigoplus_{i \in \ZZ} \Fil^i X^i \subset R[X^{\pm 1}]
			\\ & \cong \cdots \oplus R \oplus \cdots \oplus R \oplus \Fil^1R \oplus \Fil^2R \oplus \cdots 
		\end{align*}
		(This can be seen as a graded ring, with $X$ in degree $1$, as instance. But be careful of this instance, because Lemma~\ref{L:quotient of Rees algebra} below will do something that is not very friendly with graded structure.)
		
		
		
		(Also, maybe make this into a functorial construction.)
		
		(This is essentially \cite[page 170]{eisenbud}.)
		
		From this filtration on $R$, we can define its graded ring:
		$$
		\Gr^\bullet(R) \cong \bigoplus_{i \in \ZZ} \Fil^i(R)/ \Fil^{i+1}(R).
		$$
	\end{definition}
	
	\begin{lemma}
		Check the following properties of the above setup: the filtration $\Fil^\bullet $ on $R$ satisfies the following properties (mostly because of the condition $\Fil^0R = R$)
		\begin{enumerate}
			\item Each $\Fil^iR$ is an ideal of $R$.
			\item Call the filtration \emph{exhaustive} if $\bigcap \Fil^iR = 0$. Prove that, when $\Fil^\bullet$ is exhaustive, then $R^\times \notin \Fil^1(R)$.
		\end{enumerate}
		
	\end{lemma}
	
	
	\begin{lemma}
		\label{L:quotient of Rees algebra}
		Let $\Fil^\bullet$ be a filtration on $R$ satisfying the above properties. Then we have
		\begin{eqnarray*}
			\Rees(\Fil^\bullet R) / (X^{-1}) &\cong& \Gr^\bullet(R),\\
			\Rees(\Fil^\bullet R) / (X^{-1}-1) & \cong & R.
		\end{eqnarray*}
		(At the first glance, this looks quite strange, but note that typically $X \notin \Rees(\Fil^\bullet R)$ as $1 \notin \Fil^1R$.  Yet $X^{-1} \in \Rees(\Fil^\bullet R)$.)
		
		The second one can be generalized to that, for any $a \in R^\times$, we have
		$$
		\Rees(\Fil^\bullet R) /(X^{-1}-a) \cong R.
		$$
		
		(The first isomorphism is a graded isomorphism; yet the second one is strange.)
		
		\cite[Page 170]{eisenbud}. (I just thought this is a very cute lemma to prove.)
		
		(Some of this is discussed on the second half of \cite[page 120]{matsumura-ring}.)
	\end{lemma}
	
	\begin{example}
		A very important example of above is when $I$ is an ideal of $R$, we have
		$$
		\Rees_I(R): = \bigoplus_{i \in \mathbb{Z}} I^i X^i \subseteq R[X, X^{-1}].
		$$
		Then one can define a decreasing filtration on $R$ given by
		$$
		\Fil^i_IR = I^i, \qquad i \geq 0.
		$$
		(This is the most important example of the general construction.)
		
		This is somewhat different from the blow-up algebra defined in \cite[\href{https://stacks.math.columbia.edu/tag/052Q}{Definition 052Q}]{stacks-project}.
		(which is already in Mathlib)
		
		In this case, the associated graded ring becomes 
		$$
		\Gr_I(R): = \bigoplus_{i=0}^\infty I^i/I^{i+1}
		$$
	\end{example}
	
	\begin{definition}
		Let $(R, \Fil^\bullet)$ be a filtered ring. A filtered module $M$ is an $R$-module with a filtration $\Fil^\bullet M$ such that
		$$
		\Fil^i \cdot \Fil^j M \subseteq \Fil^{i+j} M.
		$$
		
		If $M$ is just an ordinary $R$-module, then one can give it a ``naive filtration as follows:
		$$
		\Fil^i(M): = \Fil^iR \cdot M \subseteq M.
		$$
		(Warning: if $M$ is a filtered $R$-module, one has to carefully distinguish the ``naive filtration" and the given filtration on $M$.)
		
		In this case, we can define the Rees polynomial in $M$ to be
		$$
		\Rees(\Fil^\bullet M): = \bigoplus_{i\in \ZZ} \Fil^iM \cdot X^i \subseteq M[X^{\pm 1}]: = M \otimes_R R[X^{\pm 1}].
		$$
		This is a graded $\Rees(\Fil^\bullet R) $-module.
		(Also, maybe make this into a functorial construction, with respect to changing $M$.)
		
		(Show that, for the naive filtration,  there is a natural homomorphism $\Rees(\Fil^\bullet R) \otimes_R M \to \Rees(\Fil^\bullet M)$ (at each graded level, it is $I^i \otimes_R M \to I^iM$). Show that this is surjective.)
		
		Similarly, we have the graded module
		$$
		\Gr_\bullet(M)= \bigoplus_{i=0}^{\infty} \Fil^iM/\Fil^{i+1}M.
		$$
		(This $\Gr_\bullet (M)$ is a graded $\Gr_\bullet(R)$-module.)
		
		(Show that there is a natural map for a filtered $R$-module $M$: 
		\begin{equation}
			\label{E:M otimes R GrIR to GrI(M)}
			\Gr_I(R)  \otimes _R M  \to \Gr_I(M),
		\end{equation}
		which can be proved using Lemma~\ref{L:Ree(M)/X} below and quotient the map $\Rees(\Fil^\bullet R) \otimes_RM \to \Rees(\Fil^\bullet M)$ by $X$. Moreover, this map is surjective.)
	\end{definition}
	
	\begin{definition}
		Let $(R, \Fil^\bullet)$ be a filtered ring. 
		Similarly, if $M$ be an $R$-module, we can similarly define Rees module
		$$
		\Rees_I(M): = \bigoplus_{i \in \mathbb{Z}} I^iMX^i \subseteq M[X, X^{-1}]: = M \otimes_R R[X, X^{-1}].
		$$
		This is a graded $\Rees_I(R)$-module.
		(Also, maybe make this into a functorial construction, with respect to changing $M$.)
		
		(Show that there is a natural homomorphism $\Rees_I(R) \otimes_R M \to \Rees_I(M)$ (at each graded level, it is $I^i \otimes_R M \to I^iM$). Show that this is surjective.)
		
		Similarly, define a decreasing filtration on $M$ by $\Fil^i_I(M): = I^iM, i \ge 0$ and then we have
		$$
		\Gr_I(M) := \bigoplus_{i=0}^{\infty} I^iM/I^{i+1}M.
		$$
		(This $\Gr_I(M)$ is a graded $\Gr_I(R)$-module.)
		
		(Show that there is a natural map 
		\begin{equation}
			\label{E:M otimes R GrIR to GrI(M)}
			M \otimes_R \Gr_I(R) \to \Gr_I(M),
		\end{equation}
		which can be proved using Lemma~\ref{L:Ree(M)/X} below and quotient the map $\Rees_I(R) \otimes_RM \to \Rees_I(M)$ by $X$. Moreover, this map is surjective.)
		
		(Some of this should already be in Mathlib; see the folder on Artin--Rees theorem in Mathlib.)
	\end{definition}
	
	\begin{lemma}
		\label{L:Ree(M)/X}
		Similar to Lemma~\ref{L:quotient of Rees algebra}, let $I$ be an ideal of $R$, $M$ an $R$-module. Then we have
		$$
		\Rees_I(M) / (X ^ {-1}) \cong \Gr_I(M).
		$$
		For any $a \in R^\times$, we have
		$$
		\Rees_I(M)/(X^{-1}-a) \cong M.
		$$
		
	\end{lemma}
	
	
	
	\subsection{Quasi-regular sequences}
	
	
	
	\begin{lemma}
		Let $R$ be a ring, $x_1, \dots, x_n$ a sequence of elements in $R$, $I = (\bfx)$ and $M$ an $R$-module. Recall that the construction in Notation~\ref{N:construction of M[X]} gives a natural homomorphism
		$$
		M[\bfX] \to M, \qquad X_i \mapsto x_i.
		$$
		which induces a natural homomorphism
		$$
		(M/IM)[\bfX] \to \Gr_I(M).
		$$
		Moreover, this map is always surjective.
		
		\cite[end of page 17]{BH}; \cite[(15.B)]{matsumura-alg}; \cite[\href{https://stacks.math.columbia.edu/tag/061N}{Equation 061N}]{stacks-project}.
	\end{lemma}
	\begin{proof}
		First, we understand $R[\bfX] \to \Gr_I(R)$ to show $(R/I)[\bfX] \to \Gr_I(R)$. For this, essentially, we use $(R/I)[\bfX] \cong R[\bfX]/I[\bfX]$ and then show that $I[\bfX]$ belongs to the kernel of the natura map $R[\bfX] \to \Gr_I(R)$.
		
		
		
		
		Next, we tensor the above with $M$ to give the needed homomorphism, well, not quite. Tensoring the above gives the first map above, then we use \eqref{E:M otimes R GrIR to GrI(M)}
		$$
		R/I[X] \otimes_R M \to \Gr_I(R)\otimes_RM \xrightarrow{\eqref{E:M otimes R GrIR to GrI(M)}} \Gr_I(M).
		$$
	\end{proof}
	
	
	
	
	\begin{definition}
		Let $R$ be a ring, $\bfx = (x_1, \dots, x_n)$ a sequence of elements in $R$, and $M$ an $R$-module. We say that $\bfx$ is {\bf $M$-quasi-sequence}  (or $M$-quasi-regular sequence) if the natural map
		$$
		(M/IM)[\bfX] \to \Gr_I(M)
		$$
		is an isomorphism.
		
		
		If $M=R$, $\bfx$ is called a {\bf quasi-regular sequence} in $R$.
		\cite[\href{https://stacks.math.columbia.edu/tag/061P}{Definition 061P}]{stacks-project}.
	\end{definition}
	
	\begin{remark}
		Alternative definitions: in the literature \cite{matsumura-ring} and \cite[(15.B)]{matsumura-alg}, this is stated with equivalent forms:
		\begin{enumerate}
			\item Write $I= (x_1, \dots,x_n)$. We say $x_1, \dots, x_n$ is an $M$-quasi-regular sequence if  for every $i >0$ and for every homogeneous polynomial $f(\bfX) \in M[X_1, \dots, X_n]$ of degree $i$ such that $F(x_1,\dots, x_n) \in I^{i+1}M$, we have $F \in IM[\bfX]$.
			\item If $F(\bfX) \in M[X_1, \dots, X_n]$ is homogeneous and $F(x_1, \dots, x_n) =0$, then the coefficients of $F$ are in $IM$.
		\end{enumerate}
	\end{remark}
	
	\begin{lemma}
		\label{L:lemma in regular -> quasireg}
		Let $R$ be a ring, $M$ an $R$-module, $\bfx = (x_1, \dots, x_n)$ an $M$-quasi-regular sequence, and $J = (x_1, \dots, x_{n})$. Suppose that $x \in A$ is an element, regular for $M/JM$.   Then $x$ is regular on $M/J^jM$ for any $j\geq 1$.
	\end{lemma}
	\begin{proof}
		Suppose that, for some $y \in M$, we have $x y \in J^j M$ for some $j \geq 1$, and we want to show that $y \in J^jM$. Arguing by induction on $j$ (base case $j=1$ is assumed), we may assume that $y \in J^{j-1} M$. So $y = G(x_1, \dots, x_n)$ for some homogeneous polynomial $G \in M[X_1, \dots, X_n]$ of degree $j-1$. Set $G' = x G$. Then the quasi-regular property applied to $G' \in M[X_1, \dots, X_n]$ yields that the coefficients of $G'$ are in $JM$. Since $x$ is regular modulo $JM$, it follows that the coefficients of $G$ are in $JM$ too, and therefore $y \in J^j M$.
	\end{proof}
	
	
	\begin{theorem}
		Let $R$ be a ring, $M$ an $R$-module, $\bfx = (x_1, \dots, x_n)$ an $M$-regular sequence, and $I = (x_1, \dots, x_n)$. Let $\bfX = (X_1, \dots, X_n)$ be indeterminates over $R$. If $F \in M[\bfX]$ is homogeneous of (total) degree $d$ and $F(\bfx) \in I^{d}M$, then the coefficients of $F$ are in $IM$.
		
		As a corollary, $M$-regular sequences are quasi-regular sequences.
	\end{theorem}
	
	
	
	
	\begin{proof}
		We use induction on $n$. The case $n = 1$ is easy. Let $n > 1$ and suppose that the theorem holds for regular sequences of length at most $n-1$.
		The proof of the theorem for sequences of length $n$ requires an additional induction on $d$. The case $d = 0$ is trivial. Assume that $d > 0$, and we have a homogeneous polynomial $F(\bfX) \in M[\bfX]$ of degree $d$ such that $F(\bfx) \in I^{d+1}M$. We need to show that all coefficients of $F$ belong to $IM$.
		
		First we reduce to the case in which $F(\bfx) = 0$, as opposed to just $F(\bfx) \in I^{d+1}M$. This is easy, because we can write $F(\bfx) = G(\bfx)$ for some $G$ homogeneous of degree $d+1$. Then rewrite $G = \sum_{i=1}^n X_iG_i$ with $G_i$ homogeneous of degree $d$. Set $G'_i = x_iG_i$ and $G' = \sum_{i=1}^n G'_i$. So $F - G'$ is homogeneous of degree $d$ and $(F - G')(\bfx) = 0$. Furthermore $F - G'$ has coefficients in $IM$ if and only if this holds for $F$.
		
		Now we may assume that $F(\bfx) = 0$. Then we write $F = G + X_nH$ with $G \in M[X_1, \dots, X_{n-1}]$.
		Since $F(\bfx) =0$, we have
		$$
		x_n \cdot H(x_1, \dots, x_n) = -G(x_1, \dots, x_{n-1}) \in J^d M \quad \textrm{with} \quad J = (x_1, \dots, x_{n-1}).
		$$
		Lemma~\ref{L:lemma in regular -> quasireg} above implies that $H(x_1, \dots,x_n) \in J^d M \subseteq I^d M$. By induction on $d$, the coefficients of $H$ are in $IM$.
		
		On the other hand, as $H(\bfx) \in J^dM$, we may write $H(x) = H'(x_1, \dots, x_{n-1})$ with $H' \in M[X_1, \dots, X_{n-1}]$ homogeneous of degree $d$. As
		\[(G + x_nH')(x_1, \dots, x_n) = F(x) = 0,\]
		it follows by induction on $n$ that $G + x_nH'$ has coefficients in $JM$. Since $x_nH'$ has its coefficients in $IM$, the coefficients of $G$ must be in $IM$ too.
	\end{proof}
	
	
	\begin{lemma}
		Prove that $x_1, \dots, x_n$ is $M$-quasi-regular if and only if $\bar x_1, \dots, \bar x_n\in I/I^2$ is a $\Gr_I(M)$-regular sequence, where $I = (x_1, \dots, x_n)$.
		
		\cite[Exercise 1.1.14]{BH}.
	\end{lemma}
	
	\begin{lemma}
		Suppose that $\bfx$ is $M$-quasi-regular, and let $I = (x_1, \dots, x_n)$. Prove:
		
		(a) If $xz \in I^i M$ for $z \in M$, then $z \in I^{i-1} M$.
		
		(b) $x_2, \dots, x_n$ is $(M/x_1 M)$-quasi-regular.
		
		(c) If $R$ is Noetherian local and $M$ is finite, then $\bfx$ is an $M$-sequence.
		
		
		\cite[Exercise 1.1.15]{BH}.
	\end{lemma}
	
	\section{Grade and depth}
	
	\subsection{Some homological algebra preparation for depth}
	
	
	\begin{proposition}
		Let $R$ be a Noetherian ring, and $M$ a finite $R$-module. If an ideal $I \subset R$ consists of zero-divisors of $M$, then $I \subset \mathfrak{p}$ for some $\mathfrak{p} \in \Ass(M)$.
		
		\cite[Proposition 1.2.1]{BH}.
	\end{proposition}
	\begin{proof}
		Since the set of zero-divisors is the union of all associated primes of $M$, so $I \subseteq \bigcup_{\frakp \in \Ass(M)}\frakp$. By a standard lemma, $I \subseteq \frakp$ for some $\frakp \in \Ass(M)$. (Here we need finiteness of $\Ass(M)$.)
	\end{proof}
	
	\begin{lemma}
		\label{L:hom(N,M)}
		Let $R$ be a ring, and $M$, $N$ be $R$-modules. Set $I = \Ann(N)$.
		
		(a) If $I$ contains an $M$-regular element, then $\Hom_R(N, M) = 0$.
		
		(b) Conversely, if $R$ is Noetherian, and $M$, $N$ are finite, then $\Hom_R(N, M) = 0$ implies that $I$ contains an $M$-regular element.
		
		\cite[Proposition 1.2.3]{BH}.
		(Check if this is already implemented.)
	\end{lemma}
	\begin{proof}
		
	\end{proof}
	
	\begin{lemma}
		\label{L:cohomology shifting technique}
		Let $R$ be a ring, $M$, $N$ be $R$-modules, and $\bfx = x_1, \dots, x_n$ a weak $M$-sequence in $\Ann(N)$. Then
		\[
		\Hom_R(N, M/\bfx M) \cong \Ext^n_R(N, M).
		\]
	\end{lemma}
	\begin{proof}
		This is done by induction on $n$. Suppose the theorem is proved with smaller $n$.
		
		Since $x_1$ is $M$-weak regular, we have
		$$
		0 \to M \xrightarrow{x_1 \cdot }M \to M/ x_1M \to 0
		$$
		Applying $\Ext^\bullet(N,-)$ to this gives
		$$
		0 \to \Hom_R(N, M) \xrightarrow{x_1\cdot }\Hom_R(N,M) \to \Hom_R(N, M/x_1M) \to \Ext^1_R(N, M) \xrightarrow{x_1 \cdot} \Ext^1_R(N, M) \to \cdots
		$$
		Note that $x_1\in \Ann(N)$, so the map $\Ext^i_R(N, M) \xrightarrow{x_1 \cdot} \Ext^i_R(N, M)$ are all zero maps. 
		
		Moreover, by induction, $\Ext_R^{n-1}(N, M) \cong \Hom_R(N, M/(x_1, \dots, x_{n-1})M)$, and $x_n$ is regular on $M/(x_1, \dots, x_{n-1})M$. Now by Lemma~\ref{L:hom(N,M)}(1), $\Hom_R(N, M/(x_1, \dots, x_{n-1})M) = 0$. So if we look at the $i=n-1$ and $n$ terms of the above long exact sequence, we get
		$$
		\begin{tikzcd}[row sep = small]
			\Ext_R^{n-1}(N, M) \ar[d, equal] \ar[r, "x_1\cdot"]&
			\Ext_R^{n-1}(N, M) \ar[d, equal] \ar[r]&
			\Ext_R^{n-1}(N, M/x_1M)  \ar[r] &
			\Ext_R^n(N, M) \ar[r, "x_1\cdot "]
			&\Ext_R^n(N, M)
			\\ 0 & 0
		\end{tikzcd}
		$$
		From this, we deduce that $\Ext^N_R(N, M) \cong \Ext^{n-1}_R(N, M/x_1M)$. The induction is complete.
	\end{proof}
	
	
	\subsection{Definition of depth}
	\begin{lemma}
		Let $R$ be a Noetherian ring and $M$ an $R$-module. If $\bfx = x_1, \dots, x_n$ is an $M$-sequence, then the sequence $(x_1) \subsetneq (x_1, x_2) \subsetneq \cdots \subsetneq (x_1, \dots, x_n)$ is strictly increasing; the sequence $x_1M$, $(x_1, x_2)M$, $\dots$ is strictly increasing as well.
	\end{lemma}
	
	
	\begin{theorem}
		Let $A$ be a noetherian ring, $M$ a finite $A$-module, and $I$ an ideal of $A$ with $IM \neq M$. Let $n >0$ be an integer. Then the following are equivalent.
		\begin{enumerate}
			\item $\Ext^i_A(N,M)=0$ for all $i<n$, and every finite $A$-module $N$ with $\supp(N) \subseteq V(I)$;
			\item $\Ext_A^i(A/I, M) =0$ for all $i<n$;
			\item there exists a finite $A$-module $N$ with $\supp(N) = V(I)$ such that $\Ext^i_A(N, M) =0 $ for every $i<n$;
			\item there exists an $M$-regular sequence $x_1, \dots, x_n$ of length $n$ in $I$.
		\end{enumerate}
		
		\cite[Theorem 28, page 101]{matsumura-alg}.
	\end{theorem}
	\begin{proof}
		$(4)\Rightarrow (1)$ is proved in Lemma~\ref{L:cohomology shifting technique}.
		$(1) \Rightarrow (2) \Rightarrow (3)$ is trivial.
		
		We now prove $(3) \Rightarrow (4)$.
	\end{proof}
	
	\begin{definition}
		Let $I$ be an ideal of a noetherian ring $A$, and $M$ a finite $A$-module.
		We define \emph{the $I$-depth} of $M$ to be the following.
		\begin{enumerate}
			\item $\depth_I(M) = \min\big\{ i\,\big|\, \Ext_A^i(A/I, M)\neq 0\big\}$;
			\item the length of {\bf any} maximal $M$-regular sequence in $I$.
		\end{enumerate}
		
		(This is the tricky point: take which definition to be the definition of depth; or depth $\leq ?$)
		
		An important special case of the definition: when $(A, \frakm)$ is a noetherian local ring and $M$ a finite $A$-module, we write $\depth(M)$ to mean $\depth_{\frakm}(M)$.
	\end{definition}
	
	\begin{lemma}
		Let $(A, \frakm)$ be a noetherian local ring, we have the following
		\begin{enumerate}
			\item $\depth M=0$ iff $\frakm \in \Ass(M)$.
		\end{enumerate}
	\end{lemma}
	
	\begin{lemma}
		If $A$ is a noetherian ring and $\frakp\subset A$ a prime ideal, we have
		$$
		\depth M_\frakp = 0 \Leftrightarrow \frakp A_\frakp \in \Ass_{A_\frakp}(M_\frakp) \Leftrightarrow \frakp \in \Ass_A(M) \Rightarrow \depth_\frakp(M) =0.
		$$
		
		In general, we have $\depth_{A_\frakp}(M_\frakp) \geq \depth_\frakp(M)$ (because localization preserves exactness).
	\end{lemma}
	
	
	\begin{theorem}
		Let $(A, \frakm)$ be a noetherian local ring and let $M \neq 0$ be a finite $A$-module. Then for every $\frakp \in \Ass(M)$, we have
		$$
		\depth M \leq \dim(A/\frakp)
		.$$
		
		\cite[Thoerem 29, page 104]{matsumura-alg}.
	\end{theorem}
	
	\section{Depth and projective dimension}
	
	
	
	\begin{thebibliography}{99}
		
		\bibitem[BH]{BH}
		Bruns and Herzog, Cohen--Macaulay rings.
		
		\bibitem[Eis]{eisenbud}
		Eisenbud, Introduction to commutative algebra
		
		\bibitem[MatsComm]{matsumura-alg}
		Matsumura, Commutative algebras.
		
		\bibitem[MatsRing]{matsumura-ring}
		Matsumura, Commutative rings.
		
		\bibitem[Stacks]{stacks-project}
		The stacks project. https://stacks.math.columbia.edu.
		
	\end{thebibliography}
	
\end{document}
